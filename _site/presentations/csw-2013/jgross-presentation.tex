\documentclass{beamer}
\usepackage{hyperref}
\usepackage{multicol}
\title[HITs]{Computational Higher Inductive Types}
\subtitle{Computing with Custom Equalities}
\author[Jason Gross]{Jason Gross\texorpdfstring{ \\ \texttt{\href{mailto:jgross@mit.edu}{jgross@mit.edu}}}{}}
\institute{MIT CSAIL Student Workshop}

\begin{document}
\begin{frame}
  \maketitle
\end{frame}

\begin{frame}
  \tableofcontents
\end{frame}

\section{Properties of Equality}
\begin{frame}{Properties of Equality}
  \pause
  \begin{itemize}
    \item Reflexivity: $x = x$ \pause
    \item Symmetry: if $x = y$ then $y = x$ \pause
    \item Transitivity: if $x = y$ and $y = z$, then $x = z$ \pause
    \item Leibniz rule: if $x = y$, then $f(x) = f(y)$
  \end{itemize}
\end{frame}

\section{Warm Up: Linked Lists}
\begin{frame}{Warm Up: Linked Lists}
  \begin{itemize}
    \item Two constructors: \pause \texttt{nil}, or \texttt{[]}, and \texttt{cons} \pause
    \item Two accessors on non-nil lists: \pause \texttt{head} and \texttt{tail} \pause
    \item Equality is defined on an element-by-element basis
       \begin{itemize}
         \item $[] = []$
         \item $[] \ne [a,\ldots]$
         \item $[a,\ldots] \ne []$
         \item $[x_0,x_1,\ldots, x_n] = [y_0,y_1,\ldots,y_m]$ iff $[x_1,\ldots,x_n] = [y_1,\ldots,y_m]$ and $x_0 = y_0$
       \end{itemize} \pause
     \item Fairly easy to prove the properties of equality
       \begin{itemize}
         \item In Coq, Agda, and Idris, you get all of these properties for free
       \end{itemize}
   \end{itemize}
\end{frame}

\section{Example: Unordered Sets}
\begin{frame}{Example: Unordered Sets}
  \pause
  \begin{itemize}
    \item \texttt{nil}, or $\emptyset$ \pause
    \item \texttt{add} \pause
    \item \texttt{remove} \pause
    \item \texttt{contains} \pause
    \item Often implemented internally as a list or a tree \pause
    \item Equality is then implemented as ``is one a permutation of the other?'' \pause
    \item Fairly easy to prove that it's an equivalence relation \pause
    \item Leibniz rule (if $x = y$, then $f(x) = f(y)$) is harder
    \item In Haskell, Agda, Coq, and Idris, the Leibniz rule is false! \pause (or at least not internally provable) \pause
      \begin{itemize}
        \item The problem is that either you don't have private
          fields, or you can't make use of the fact that everything is
          defined in terms of your public methods.
      \end{itemize}
  \end{itemize}
\end{frame}

\subsection{Canonical Inhabitants}
\begin{frame}{Example: Unordered Sets}{Solution 1: Canonical Inhabitants}
  \begin{itemize}
    \item Give up private fields, but use element-wise equality \pause
    \item Define a type of ``sorted lists without duplication'', and call them sets \pause
    \item Now we can use element-wise equality, and get Leibniz (and other properties) for free \pause
    \item What if we don't have an ordering on the elements, only equality? \pause
    \item Is this really what we wanted?  We asked for unordered sets, and instead made sorted lists.
  \end{itemize}
\end{frame}

\subsection{Higher Inductive Types}
\begin{frame}{Example: Unordered Sets}{Solution 2: Higher Inductive Types}
  \begin{itemize}
    \item Higher Inductive Types \pause
    \item Keep the built-in equality (so we get the properties for free), but turn it into equality up to permutation \pause
    \item How do we get that it's an equivalence relation for free? \pause
      \begin{itemize}
        \item Take the reflexive symmetric transitive closure of the given relation
      \end{itemize} \pause
    \item How do we get Leibniz for free? \pause
      \begin{itemize}
        \item Require proving it each time you define a particular function
        \item To define a function that deals with unordered sets,
          you have to simultaneously prove that your function is
          invariant under permutations
      \end{itemize}
  \end{itemize}
\end{frame}

\section{Computing with Higher Inductive Types}
\begin{frame}{Computing with Higher Inductive Types}
  \begin{itemize}
    \item It seems simple enough, so what's the problem? \pause
    \item Having higher inductive types gives you functional extensionality (if $f(x) = g(x)$ for all $x$, then $f = g$), which doesn't yet have a good computational interpretation in Coq nor Agda nor Idris \pause
    \item Equality in Coq and Agda (\texttt{--without-K}) actually has a rich structure \pause
    \item If you look at proofs of equality, and equality of these proofs, and you iterate this process, you get enough math to do topology! \pause
    \item This is Homotopy Type Theory
  \end{itemize}
\end{frame}


\section{Thank you}
\begin{frame}{Thank you}
  \noindent \Huge
  \noindent \begin{center}
    \noindent Thanks! \\ $\left.\right.$ \pause \\
    \noindent Questions?
  \end{center}
\end{frame}

\section*{Parametricity}
\begin{frame}{Example: Unordered Sets}{Solution 3: Parametricity}
  \begin{itemize}
    \item Make use of the fact that private fields are private \pause
    \item Very hard to do! \pause
    \item Can probably be done by way of parametricity (aka ``theorems for free''), or a generalization of it \pause
    \item Parametricity can be given a computational interpretation,
      but it's very non-trivial to do so
  \end{itemize}
\end{frame}



\end{document}
