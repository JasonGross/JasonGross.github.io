\documentclass[utf8x]{beamer}

\usepackage{beamerthemesplit}
\usepackage{url}
\usepackage{amsmath}
\usepackage{amssymb}
\usepackage{verbatim}
\usepackage{hyperref}
\usepackage{pgf,tikz}
\usetikzlibrary{arrows}
\usepackage[all]{xy}


\makeatletter
\AtBeginDocument{% example environment modified from the example environment in lshort.sty, using the verbatim package
  \newwrite\examplesx@out
  \newenvironment{examplesx}{%
    \begingroup% Lets Keep the Changes Local
      \@bsphack
      \immediate\openout \examplesx@out \jobname.exa
      \let\do\@makeother\dospecials\catcode`\^^M\active
      \def\verbatim@processline{%
        \immediate\write\examplesx@out{\the\verbatim@line}}%
      \verbatim@start
    }{%
    \immediate\closeout\examplesx@out\@esphack\endgroup%
    \noindent\makebox[\textwidth][l]{%
      \begin{minipage}[c]{0.45\textwidth}%
        \small\verbatiminput{\jobname.exa}
      \end{minipage}%
      \hspace*{0.1\textwidth}%
      \framebox{%
        \begin{minipage}{0.45\textwidth}%
          \small\input{\jobname.exa}%
        \end{minipage}
      }%
    }\vspace*{\parskip}%
  }
}
\makeatother

\pdfinfo {
  /Title (Database Management on top of Category Theory in Coq)
  /Subject (POPL)
  /Author (Jason Gross)
  /Keywords (POPL, Coq, Category Theory, Databases)
}

\begin{document}

\title[Database Management on top of Category Theory in Coq]{Building Database Management on top of Category Theory in Coq}
\date{POPL 2013}
\author{Jason Gross --- \href{mailto:jgross@mit.edu}{jgross@mit.edu}}

\begin{frame}
  \titlepage
  This document is available at \url{http://web.mit.edu/jgross/Public/POPL/\jobname.pdf}.

  My category theory library is available at \url{https://bitbucket.org/JasonGross/catdb}.
\end{frame}

\section*{Outline}
  \begin{frame}{Outline}
    \tableofcontents[pausesections]
  \end{frame}

\section{Introduction --- Databases and Category Theory}
  \subsection{Categories}
    \begin{frame}{Categories}
      A category is:
      \begin{itemize}
        \item
          a collection of objects, \pause together with
        \item
          arrows between those objects, \pause together with
        \item
          a composition law for the arrows satisfying coherence conditions:
          \begin{itemize}
            \item existence of identity
            \item associativity
          \end{itemize} \pause
      \end{itemize}
      \begin{center}
        \fbox{
          $$
          \xymatrix{
            a \ar@/^{0.5pc}/[rr]^{f} \ar@/_{0.5pc}/[d]_{g} \ar@/^{0.5pc}/[d]^{h} && b \ar@/^{0.5pc}/[ll]^{i} \ar@/^{1.5pc}/[dll]^{j} \\
            c
          }
          $$
        }
      \end{center}
    \end{frame}

  \subsection{Relational Databases}
    \begin{frame}{Relational Databases}
      A database schema for a relational database can be modeled as a
      \begin{itemize}
        \item collection of \textbf{tables}, \pause together with
        \item a collection of \textbf{attributes} or column-labels for each table, \pause together with
        \item integrity constraints \pause
      \end{itemize}
      \begin{center}
        \fbox{
          $$
          \xymatrix{
            *\txt{professor} \ar@/^{1pc}/[rr]^{\text{department}} \ar@/_{1pc}/[dd]_{\text{first name}} \ar@/^{1pc}/[dd]^{\text{last name}} && *\txt{department} \ar@/^{1pc}/[ll]^{\text{department head}} \ar@/^{2pc}/[ddll]^{\text{name}} \\ \\
            *\txt{string} &&
          }
          $$
          }
      \end{center}
    \end{frame}

  \subsection{Relational Database Schema = Category}
    \begin{frame}{Relational Database Schema = Category}
      \begin{center}
        \only<1,3>{
          \fbox{
          $$
          \xymatrix{
            a \ar@/^{0.5pc}/[rr]^{f} \ar@/_{0.5pc}/[d]_{g} \ar@/^{0.5pc}/[d]^{h} && b \ar@/^{0.5pc}/[ll]^{i} \ar@/^{1.5pc}/[dll]^{j} \\
            c
          }
          $$
          }
        }
        \only<3>{\raisebox{-3.5ex}[\height][\depth]{$\cong$}}
        \only<2,3>{
          \fbox{\only<3>{\small}
          $$
          \xymatrix{
            *\txt{professor} \ar@/^{1pc}/[rr]^{\text{department}} \ar@/_{1pc}/[dd]_{\text{f.~name}} \ar@/^{1pc}/[dd]^{\text{l.~name}} && *\txt{department} \ar@/^{1pc}/[ll]^{\text{department head}} \ar@/^{2pc}/[ddll]^{\text{name}} \\ \\
            *\txt{string} &&
          }
          $$
          }
        }
        \only<3>{\vspace{4ex} \\ The diagrams are ``the same''.}
      \end{center}
    \end{frame}

  \subsection{Usefulness}
    \begin{frame}{Usefulness of Categorical Databases}
      \begin{itemize}
        \item
          Built in notion of path equivalence (multiple equivalent paths of foreign keys can be a pain in typical database management). \pause
        \item
          Provides a rigorous language for data migration between databases (another hard task in standard database management).
      \end{itemize}
    \end{frame}

\section{Category Theory in Coq}

  \begin{frame}{Category Theory in Coq}
    \begin{itemize}
      \item
        Many people learn a proof assistant by coding up category theory. \pause
      \item
        Category theory is relatively simple to code up. \pause
        \begin{itemize}
          \item
            Standard rigorous formulation of concepts exists in the literature. \pause
          \item
            It's rare to get caught up in minute details of proofs. \pause
          \item
            If you can define something categorically, it's probably interesting.
        \end{itemize}
    \end{itemize}
  \end{frame}

  \subsection{Universe Levels}
    \begin{frame}
      \frametitle{Universe Levels \only<1-7>{(Russel's Paradox)}}
      \begin{itemize}
        \item<1-7>
          Consider, na\"ively, the set of all sets. \only<2->{Does it contain itself?} \only<3-7>{
          \begin{itemize}
            \item
              It's a set, and it contains all sets, so it must.
          \end{itemize}
        }
        \item<4-7>
          Consider the set of all sets that do not contain themselves.  Does it contain itself?
          \only<5-7>{
          \begin{itemize}
            \item<5->
              If it contains itself, then it is not a set that doesn't contain itself, and so it cannot be a member of itself; contradiction.  Thus it cannot contain itself. \pause
            \item<6->
              If it does not contain itself, then it is a set that does not contain itself, and thus must be a member of itself; contradiction.  Thus it cannot fail to contain itself.
          \end{itemize}
        }
        \item<7->
          This is the \textbf{paradox of na\"ive set theory}.
        \item<8->
          Solution: universe levels
          \begin{itemize}
            \item<9->
              \texttt{Set} or \texttt{Type}(0) is the collection of all sets, \texttt{Type}(1) is the collection of all \texttt{Type}(0)s, \ldots, \texttt{Type}($i + 1$) is the collection of all \text{Type}($i$)s
            \item<9->
              The \textbf{universe level} of an object of type \texttt{Type}($i$) is $i$
          \end{itemize}
        \item<10->
          In some cases, Coq can infer the universe level of an inductive type from the universe levels of its parameters; when this happens, the inductive type is polymorphic over universe levels.
        \item<11->
          It's useful to talk about ``a category whose objects are of type \texttt{T}'' rather than just ``a category''.
      \end{itemize}
    \end{frame}

  \subsection{Limits and Colimits}
    \begin{frame}{Limits and Colimits}
      \begin{itemize}
        \item
          Categorical \textbf{limits} are like \textbf{Cartesian products}, subject to constraints about equality of components \pause
        \item
          Categorical \textbf{colimits} are like \textbf{disjoint unions}, modulo equivalence relations
      \end{itemize}
    \end{frame}

    \begin{frame}{Coq Category}
      \begin{itemize}
        \item Coq has all limits
          \begin{itemize}
            \item
              Product types provide products (function types, e.g, \texttt{forall~a~:~A,~f~a} is the product $\prod_{a \in A} f(a)$)
            \item
              Sigma types provide constraints about equality of components (e.g., \texttt{\{ f : A $\to$ B | f a = f b \}})
          \end{itemize} \pause
        \item Coq has some colimits
          \begin{itemize}
            \item
              Sigma types provide disjoint unions (e.g., \texttt{\{ j : J \& f j \}} is the disjoint union $\bigsqcup_{j \in J} f(j)$)
            \item
              Quotients are \ldots\space hard
          \end{itemize}
      \end{itemize}
    \end{frame}

    \begin{frame}{Quotients}
      \begin{itemize}
        \item Quotients can be defined via axioms \pause
        \begin{itemize}
          \item
            \texttt{proof\_irrelevance}; \texttt{(A $\leftrightarrow$ B) $\to$ A = B} for propositions; either decidable existence, or a way of turning proofs of existence into objects (\texttt{constructive\_indefinite\_description : (exists~x,~P~x)~$\to$~\{~x~|~P~x~\}})
          \item
            Not computational
        \end{itemize} \pause
        \item Quotients can be defined via setoids \pause
        \begin{itemize}
          \item
            All objects carry around extra information of what the equivalence relation is
          \item
            This is somewhat clunky
          \item
            Not first-class quotients
        \end{itemize}
      \end{itemize}
    \end{frame}

    \begin{frame}{Limits and Colimits (High-Level Summary)}
      \begin{itemize}
        \item
          There are two categorical constructions (limits and colimits) that are ``dual'' \pause
        \item
          Coq's type-system fully implements only one of these (limits) \pause
        \item
          It's harder to define colimits inside of Coq than limits, in general, even for the ones that Coq does support
      \end{itemize}
    \end{frame}

  \begin{frame}[Thank You]
    \fontsize{128}{0}\selectfont\begin{center}Thank You!\end{center}
  \end{frame}
\end{document}
